\documentclass{article}
\usepackage[margin=0.5in]{geometry}
\usepackage{titlesec}
\usepackage{ifthen}
\usepackage{fancyhdr}
\usepackage{xcolor}
\usepackage{mathtools}
\usepackage{amsmath}
\usepackage{dsfont}
\usepackage{enumitem}
\usepackage{listings}

\DeclarePairedDelimiter\bra{\langle}{\rvert}
\DeclarePairedDelimiter\ket{\lvert}{\rangle}
\DeclarePairedDelimiterX\braket[2]{\langle}{\rangle}{#1 \delimsize\vert #2}

% -------- %
% SECTIONS %
% -------- %
\newcounter{problemnumber}\setcounter{problemnumber}{1}
\titlespacing\section{0pt}{10pt}{0pt}   % Spacing between Problems
\titlespacing\subsection{0pt}{5pt}{0pt} % Spacing between Parts
\newcommand{\problem}[1][-1]{
    \setcounter{partnumber}{1}
    \ifnum#1>0
        \setcounter{problemnumber}{#1}
    \fi
    \section*{Problem \arabic{problemnumber}}
    \stepcounter{problemnumber}
}

\newcounter{partnumber}\setcounter{partnumber}{1}
\newcommand{\ppart}[1][-1]{
    \ifnum#1>0
        \setcounter{partnumber}{#1}
    \fi
    \subsection*{Part \parttype{partnumber}}
    \stepcounter{partnumber}
}

\newenvironment{question}{
    \color{gray}\itshape
    \vspace{5pt}
    \begin{tabular}{|p{0.97\linewidth}}
}{
    \end{tabular}\\[5pt]
}



% ------------- %
% HEADER/FOOTER %
% ------------- %
\setlength\parindent{0pt}
\setlength\headheight{30pt}
\headsep=0.25in
\pagestyle{fancy}
\lhead{\ifthenelse{\thepage=1}
    {\textbf{Trevor Smith} \\ \textbf{\writeday}}
}
\chead{\ifthenelse{\thepage=1}
    {\textbf{\huge{HOMEWORK \hwnumber}}}
    {\textbf{\large{HOMEWORK \hwnumber}}}
}
\rhead{\ifthenelse{\thepage=1}
    {\textbf{{\course}} \\ \textbf{Professor {\prof}}}
}
\cfoot{\thepage}
\renewcommand\headrulewidth{0.4pt}
\renewcommand\footrulewidth{0.4pt}



% ---------- %
% PARAMETERS %
% ---------- %
% \PARTTYPE:
% \Alph   := "Part A, Part B,  ..."
% \alph   := "Part a, Part b,  ..."
% \arabic := "Part 1, Part 2,  ..."
% \Roman  := "Part I, Part II, ..."
\newcommand\parttype{\Roman}

% \COURSE:
\newcommand\course{EECE 2322}

% \HWNUMBER
\newcommand\hwnumber{6}

% \SEMESTER
\newcommand\semester{Spring 2022}

% \PROF
\newcommand\prof{Consi}

% \WRITEDAY
% \today is date of compilation, replace if writing due date rather than write date
\newcommand\writeday{\today}



%  ------- %
% DOCUMENT %
% -------- %
\begin{document}
\problem

\begin{itemize}
	\item $0x00134880 = 0b000000\ 00000\ 10011\ 01001\ 00010\ 000000 $ \\
		op = 0, rs = 0, rt = 19, rd = 9, shamt = 2, funct = 0 \\
		\begin{lstlisting} 
sll $t1, $s3, 2  # Shift 0 left 2 and store it in reg t1
		\end{lstlisting}

	\item $ 0x01364820 = 0b0000\ 0001\ 0011\ 0110\ 0100\ 1000\ 0010\ 0000 $ \\
		$ = 0b000000\ 01001\ 10110\ 01001\ 00000\ 100000 $ \\
		op = 0, rs = 9, rt = 22, rd = 9, shamt = 0, funct = 32 \\
		\begin{lstlisting} 
add $t1, $t1, $s6  # t1 = t1 + s6
		\end{lstlisting}

	\item $0x8D280000 = 0b1000\ 1101\ 0010\ 1000\ 0000\ 0000\ 0000\ 0000  $ \\
		$ = 0b100011\ 01001\ 00000\ 0000\ 0000\ 0000\ 0000  $ \\
		op = 35, rs = 9, rt = 8, imm = 0 \\
		\begin{lstlisting} 
lw $t0, 0($t1)
		\end{lstlisting}

	\item $ 0x15150002 = 0b0001\ 0101\ 0001\ 0101\ 0000\ 0000\ 0000\ 0010 $
		$ = 0b000101\ 01000\ 10101\ 0000\ 0000\ 0000\ 0010 $
		op = 5, rs = 8, rt = 21, imm = 2
		\begin{lstlisting} 
bne $t0, $s5, 2  # Jumps to the end of the file
		\end{lstlisting}

	\item $0x22730001 = 0b0010\ 0010\ 0111\ 0011\ 0000\ 0000\ 0000\ 0001 $
		$ = 0b001000\ 10011\ 10011\ 0000\ 0000\ 0000\ 0001 $
		op = 8, rs = 19, rt = 19, imm = 1
		\begin{lstlisting} 
addi $s3, $s3, 1  # increments s3 (looping var)
		\end{lstlisting}

	\item $ 0x08100004 = 0b0000\ 1000\ 0001\ 0000\ 0000\ 0000\ 0000\ 0100 $
		$ = 0b000010\ 00\ 0001\ 0000\ 0000\ 0000\ 0000\ 0100 $
		op = 2, addr = 0x100004
		\begin{lstlisting} 
jmp 0100004  # jumps to sll
		\end{lstlisting}
\end{itemize}

Overall, this code loops on the array until it hits a value that isn't 5.
Array accessing is done on the looping var s3, which is bit-shifted by 2
bits (multiplying by 4) to get the address of each 32-bit word in the array.

At the end of the program, s3 is 4, t0 is 6, t1 is the array address + 4xs3 or
10010010, s6 is 10010000, s5 is still 5. 

\newpage
\problem

Approach here is to represent the actual resistor values in an array, and to
take advantage of the fact that they are already listed in increasing order.
We will therefore \\
\begin{itemize}
	\item take an input val, s0
	\item vet the input val, if not go back
	\item iterate with array index var, s1 (inc by 4)
	\item store array val in s2
	\item old array item is in t2, initially it's zero
	\item if new array item is larger than our input value, we need to
		decide which is closer
	\item take differences of each and store them in t3 (smaller val) and
		t4 (larger val)
	\item t3 = s0 - t2
	\item t4 = s2 - s0
	\item if t3 \< t4, our answer is t2, otherwise it's s2
\end{itemize}
(This design was used as a reference for writing the code. Functionally,
the code is the same.) \\

Correct code is attached. A sample output from the program can be found
below (unedited) since attaching screenshots is slightly inconvenient
in a latex file and this does just as well.

\begin{lstlisting}
Please enter desired resistance or 0 to quit.
-134
Input cannot be negative.
Please enter desired resistance or 0 to quit.
12312321
Input cannot be larger than the largest resistor.
Please enter desired resistance or 0 to quit.
12123
You should use the resistor marked with resistance: 12000
-- program is finished running (dropped off bottom) --


Reset: reset completed.

Please enter desired resistance or 0 to quit.
14123
You should use the resistor marked with resistance: 15000
-- program is finished running (dropped off bottom) --


Reset: reset completed.

Please enter desired resistance or 0 to quit.
0
Quitting program.
\end{lstlisting}

As you can see, I said `correct code' because it performs
exactly as specified. One problem I had before it was
done, is it would always print the larger value. This
ended up being because I hadn't put in the logic yet
for checking which one was closer (just stored the
difference between lower and higher number and the
desired resistance but didn't do anything with them).
Once I added that it worked. This is demonstrated by
the sample output - two numbers 12000 and 15000 are
one after another in the array. With the input 12123,
clearly closer to 12000, the program gave 12000.
With the input 14123, the program gave 15000.\\

Future versions of this program could improve on the concept
by using an arbitrary number of resistors in series to get close as
possible to the input value. Even further work could
guarantee an exact resistance value by using any number of
resistors and resistors in parallel or in series.

\end{document}
